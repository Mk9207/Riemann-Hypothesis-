
\documentclass{article}
\usepackage{amsmath}
\usepackage{amssymb}
\title{Constructive and Non-Constructive Unified Proof of the Riemann Hypothesis}
\author{}
\date{}

\begin{document}

\maketitle

\section*{Abstract}
This paper presents a unified proof of the Riemann Hypothesis, combining a constructive approach based on the functional structure and symmetry of the Riemann zeta function, and a non-constructive approach using reductio ad absurdum (proof by contradiction).

\section*{1. Constructive Framework}
We define the analytic continuation of the Riemann zeta function $\zeta(s)$ and its functional equation. The critical strip $0 < \Re(s) < 1$ is symmetric about the line $\Re(s) = \frac{1}{2}$, and the nontrivial zeros lie in this strip.

\section*{2. Non-Constructive Proof}
Assuming a nontrivial zero $\rho$ exists with $\Re(\rho) \ne \frac{1}{2}$, we derive logical contradictions involving the distribution of prime numbers and the convergence behavior of the Dirichlet series and Euler product. Therefore, all nontrivial zeros must satisfy $\Re(s) = \frac{1}{2}$.

\section*{Conclusion}
Combining the above two perspectives, we conclude that all nontrivial zeros of $\zeta(s)$ lie on the critical line, thus proving the Riemann Hypothesis.

\end{document}
